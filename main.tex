% !TEX TS-program = xelatex
\documentclass[a4paper]{article}
\usepackage[left=1in, right=1in, top=1in, bottom=1in]{geometry}
\usepackage{fontspec}
\usepackage{polyglossia}
\setdefaultlanguage{thai}
\setotherlanguage{english}
\usepackage{graphicx}

% Font setup for Thai with proper line breaking
\newfontfamily\thaifont{TH Sarabun New}[
  Script=Thai,
  Scale=1.0,
  WordSpace=1.0,
  LetterSpace=0.0
]
\newfontfamily\englishfont{TH Sarabun New}[
  Script=Latin,
  Scale=1.0
]
\setmainfont{TH Sarabun New}

% Enable Thai line breaking and hyphenation
\XeTeXlinebreaklocale "th"
\XeTeXlinebreakskip = 0pt plus 1pt minus 1pt

% Single spacing with proper line height for Thai
\usepackage{setspace}
\setstretch{1.2}

% 16pt body size with better line height
\makeatletter
\renewcommand\normalsize{%
  \@setfontsize\normalsize{16pt}{19.2pt}%
}
\normalsize
\makeatother

% Section titles formatting
\usepackage{titlesec}
\titleformat{\section}{\fontsize{18pt}{21.6pt}\selectfont\bfseries}{\thesection}{1em}{}
\titleformat{\subsection}{\fontsize{16pt}{19.2pt}\selectfont\bfseries}{\thesubsection}{1em}{}
\titleformat{\subsubsection}{\fontsize{16pt}{19.2pt}\selectfont\bfseries}{\thesubsubsection}{1em}{}

% TOC formatting
\usepackage{tocloft}
\renewcommand{\cftsecfont}{\fontsize{16pt}{19.2pt}\selectfont}
\renewcommand{\cftsubsecfont}{\fontsize{16pt}{19.2pt}\selectfont}
\renewcommand{\cftsecpagefont}{\fontsize{16pt}{19.2pt}\selectfont}
\renewcommand{\cftsubsecpagefont}{\fontsize{16pt}{19.2pt}\selectfont}

% Essential packages
\usepackage[hidelinks]{hyperref}
\usepackage{booktabs}
\usepackage{csquotes}

% Thai text formatting improvements for formal documents
\usepackage{ragged2e}
\justifying % Full justification for formal thesis
\hyphenpenalty=10000 % Disable hyphenation for Thai
\exhyphenpenalty=10000

% Bibliography setup for APA
\usepackage[style=apa,backend=biber]{biblatex}
\DeclareLanguageMapping{english}{english-apa}
\addbibresource{references.bib}

% Thai alphabet page numbering - complete implementation
\makeatletter
\def\@thaialph#1{%
  \ifcase#1\or ก\or ข\or ค\or ง\or จ\or ฉ\or ช\or ซ\or ฌ\or ญ\or
  ฎ\or ฏ\or ฐ\or ฑ\or ฒ\or ณ\or ด\or ต\or ถ\or ท\or ธ\or น\or บ\or ป\or ผ\or ฝ\or พ\or ฟ\or ภ\or ม\or
  ย\or ร\or ล\or ว\or ศ\or ษ\or ส\or ห\or ฬ\or อ\or ฮ\else\@ctrerr\fi
}
\def\thaialph#1{\expandafter\@thaialph\csname c@#1\endcsname}
\makeatother

% Formal Thai thesis paragraph settings
\setlength{\parindent}{2em}   % Standard Thai thesis indentation
\setlength{\parskip}{0pt}     % No space between paragraphs

% Better text justification for Thai
\tolerance=1000
\pretolerance=800
\emergencystretch=3em

% Cover page title setup
\date{}
\author{}
\title{
    \includegraphics[width=1in, height=1in, keepaspectratio]{logo.png}
    \\[2ex]
    {\fontsize{32pt}{36pt}\selectfont\textbf{Polyfunctional Robots}}
}

\begin{document}

% ----- Cover Page (no page number) -----
\maketitle
\thispagestyle{empty}

\vfill
\begin{center}
    {\fontsize{22pt}{26pt}\selectfont\textbf{
        นายคมชาญ วิเศษนคร
        \\ 663040419-1
    }}
\end{center}
\vfill

\vfill
\begin{center}
    {\fontsize{16pt}{20pt}\selectfont\textbf{
        รายงานฉบับนี้เป็นส่วนหนึ่งของรายวิชา EN813761 การสัมมนาทางวิศวกรรมคอมพิวเตอร์
        \\ สาขาวิชาวิศวกรรมคอมพิวเตอร์ มหาวิทยาลัยขอนแก่น
        \\ ภาคเรียนที่ 1 ปีการศึกษา 2568
    }}
\end{center}

\newpage

% ----- Start Thai letter page numbering for front matter -----
\pagenumbering{thaialph}
\setcounter{page}{1}

% ----- Abstract (Thai) -----
{\centering
    {\fontsize{18pt}{21.6pt}\selectfont\textbf{บทคัดย่อ}\par}
}
\vspace{1em}

% Thai abstract content goes here

\vspace{1em}

% ----- Abstract (English) -----
{\centering
    {\fontsize{18pt}{21.6pt}\selectfont\textbf{Abstract}\par}
}
\vspace{1em}

% English abstract content goes here

\vspace{1em}

\noindent
\textbf{คำสำคัญ:} หุ่นยนต์อเนกประสงค์, หุ่นยนต์โมดูลาร์, การควบคุมแบบลำดับชั้น, เซ็นเซอร์หลายโหมด, แอคชูเอเตอร์ปรับความแข็งได้

\vspace{0.5em}

\noindent
\textbf{Keywords:} Polyfunctional Robots, Modular Robotics, Hierarchical Control, Multimodal Sensors, Variable Stiffness Actuators

\newpage

% ----- Table of Contents -----
\tableofcontents
\newpage

% ----- Switch to Arabic numbering for main content -----
\pagenumbering{arabic}
\setcounter{page}{1}

% ----- Unnumbered front matter sections -----
\section*{คำนำ}
% คำนำ content goes here

\newpage

\section*{องค์ประกอบของรายงาน}
% รายละเอียดองค์ประกอบตามเกณฑ์การประเมิน

\newpage

% ----- Main Content Sections (Numbered) -----

\section{บทนำ}
หุ่นยนต์อเนกประสงค์ (Polyfunctional Robots) หรือหุ่นยนต์อเนกฟังก์ชัน เป็นระบบหุ่นยนต์ขั้นสูงที่ออกแบบให้สามารถปฏิบัติภารกิจที่หลากหลายและซับซ้อนภายในระบบเดียว โดยไม่จำเป็นต้องมีการปรับเปลี่ยนฮาร์ดแวร์หลักอย่างมีนัยสำคัญ \parencite{liang2025decoding} ซึ่งแตกต่างจากหุ่นยนต์แบบดั้งเดิมที่มักออกแบบมาเพื่อปฏิบัติงานเฉพาะทางเพียงอย่างเดียว หุ่นยนต์อเนกประสงค์สามารถปรับเปลี่ยนฟังก์ชันการทำงานผ่านการปรับโครงสร้างทางกายภาพ (Physical Reconfiguration) การเขียนโปรแกรมควบคุมใหม่ (Software Reconfiguration) หรือการผสมผสานโมดูลต่างๆ เข้าด้วยกัน \parencite{post2023modular}

ในยุคของการปฏิวัติอุตสาหกรรม 4.0 และการพัฒนาปัญญาประดิษฐ์ ความต้องการหุ่นยนต์ที่มีความยืดหยุ่นและสามารถปรับตัวได้กับสภาพแวดล้อมที่เปลี่ยนแปลงเพิ่มขึ้นอย่างรวดเร็ว \parencite{mohammadi2023mobile} หุ่นยนต์อเนกประสงค์จึงกลายเป็นทางเลือกที่มีความสำคัญสำหรับอุตสาหกรรมต่างๆ ตั้งแต่การผลิตและการประกอบชิ้นส่วน ไปจนถึงการแพทย์และการสำรวจอวกาศ เนื่องจากสามารถลดต้นทุนการลงทุนและเพิ่มประสิทธิภาพการใช้งานผ่านการใช้ระบบเดียวสำหรับหลายงาน

แนวคิดของหุ่นยนต์อเนกประสงค์มีความเชื่อมโยงอย่างใกล้ชิดกับหุ่นยนต์โมดูลาร์ (Modular Robots) และหุ่นยนต์ที่ปรับโครงสร้างได้ด้วยตนเอง (Self-Reconfiguring Robots) \parencite{seo2019modular} อย่างไรก็ตาม หุ่นยนต์อเนกประสงค์มีจุดเน้นที่แตกต่างออกไป คือ การเน้นที่ความสามารถในการปฏิบัติงานหลากหลายประเภทมากกว่าการเปลี่ยนรูปร่างหรือโครงสร้าง ทำให้เหมาะสำหรับการประยุกต์ใช้ในสภาพแวดล้อมที่ต้องการความเชี่ยวชาญในหลายด้านพร้อมกัน

\subsection{วัตถุประสงค์ของการศึกษา}

การศึกษานี้มีวัตถุประสงค์หลักเพื่อวิเคราะห์และสังเคราะห์องค์ความรู้เกี่ยวกับหุ่นยนต์อเนกประสงค์ในมุมมองทางวิศวกรรม โดยมีจุดมุ่งหมายเฉพาะ ดังนี้

\textbf{1. วิเคราะห์สถาปัตยกรรมและการออกแบบ} เพื่อศึกษาหลักการออกแบบสถาปัตยกรรมแบบโมดูลาร์ที่ปรับโครงสร้างได้ (Modular Reconfigurable Architecture) และระบบควบคุมแบบลำดับชั้น (Hierarchical Control Systems) ที่เป็นพื้นฐานสำคัญของหุ่นยนต์อเนกประสงค์ \parencite{tassi2024multimodal}

\textbf{2. ศึกษาเทคโนโลยีหลัก} โดยเฉพาะการบูรณาการเซ็นเซอร์แบบหลายโหมด (Multimodal Sensor Integration) \parencite{yang2024body} แอคชูเอเตอร์ปรับความแข็งได้ (Variable Stiffness Actuators) และการประยุกต์ใช้ปัญญาประดิษฐ์แบบโมเดลพื้นฐาน (Foundation Models) ในการควบคุม

\textbf{3. วิเคราะห์การประยุกต์ใช้} ในภาคอุตสาหกรรมสำคัญ รวมถึงการผลิตอัตโนมัติ การแพทย์ การสำรวจอวกาศ และการกู้ภัยพิบัติ พร้อมทั้งประเมินประสิทธิภาพและความคุ้มค่าทางเศรษฐศาสตร์

\textbf{4. ระบุความท้าทายและข้อจำกัด} ทั้งในด้านเทคนิคและการนำไปใช้งานจริง รวมถึงประเด็นด้านความปลอดภัยและมาตรฐานสากล เช่น ISO 10218-1:2025 \parencite{iso2025robotics}

\subsection{ขอบเขตการศึกษา}

การศึกษานี้มุ่งเน้นหุ่นยนต์อเนกประสงค์ในบริบททางวิศวกรรม โดยครอบคลุมระบบที่มีความสามารถในการปฏิบัติงานหลากหลายผ่านกลไกต่างๆ ดังนี้

\textbf{ขอบเขตด้านเทคนิค} การศึกษาครอบคลุมหุ่นยนต์ที่สามารถปรับเปลี่ยนฟังก์ชันผ่าน (1) การปรับโครงสร้างทางกายภาพแบบโมดูลาร์ (2) การเปลี่ยนแปลงอัลกอริทึมควบคุมและซอฟต์แวร์ และ (3) การผสมผสานโมดูลฮาร์ดแวร์ที่แตกต่างกัน

\textbf{ขอบเขตด้านการประยุกต์ใช้} เน้นการใช้งานในสภาพแวดล้อมอุตสาหกรรมและการค้า รวมถึงการแพทย์ การสำรวจ และการบริการ โดยไม่รวมถึงหุ่นยนต์เฉพาะทางที่ไม่สามารถปรับเปลี่ยนฟังก์ชันได้

\textbf{ขอบเขตด้านเวลา} การศึกษาเน้นงานวิจัยและพัฒนาตั้งแต่ปี ค.ศ. 2015 ถึงปัจจุบัน โดยเฉพาะอย่างยิ่งความก้าวหน้าในช่วง 5 ปีล่าสุดที่มีการประยุกต์ใช้ปัญญาประดิษฐ์และเทคโนโลยีการเรียนรู้ของเครื่อง

\subsection{คำจำกัดความสำคัญ}

เพื่อความชัดเจนในการศึกษา จึงกำหนดคำจำกัดความของแนวคิดสำคัญ ดังนี้

\textbf{หุ่นยนต์อเนกประสงค์ (Polyfunctional Robots)} หมายถึง ระบบหุ่นยนต์ที่สามารถปฏิบัติงานที่หลากหลายและแตกต่างกันได้ภายในระบบเดียว โดยมีความสามารถในการปรับเปลี่ยนฟังก์ชันการทำงานตามความต้องการของงานแต่ละประเภท

\textbf{หุ่นยนต์โมดูลาร์ (Modular Robots)} หมายถึง หุ่นยนต์ที่ประกอบด้วยโมดูลแยกส่วนที่สามารถเชื่อมต่อและแยกออกจากกันได้ เพื่อสร้างโครงสร้างและฟังก์ชันใหม่ตามต้องการ \parencite{bi2016survey}

\textbf{หุ่นยนต์ปรับโครงสร้างได้ (Self-Reconfiguring Robots)} หมายถึง หุ่นยนต์ที่สามารถเปลี่ยนแปลงรูปร่างและโครงสร้างของตนเองได้โดยอัตโนมัติ เพื่อให้เหมาะสมกับงานหรือสภาพแวดล้อมที่แตกต่างกัน \parencite{hameed2017modular}

\textbf{การควบคุมแบบลำดับชั้น (Hierarchical Control)} หมายถึง ระบบควบคุมที่จัดระดับการควบคุมเป็นชั้นๆ โดยชั้นบนมีหน้าที่วางแผนและตัดสินใจระดับสูง ส่วนชั้นล่างดำเนินการควบคุมรายละเอียดเฉพาะทาง

\subsection{ระเบียบวิธีการศึกษา}

การศึกษานี้ใช้วิธีการทบทวนวรรณกรรมเชิงพรรณนา (Descriptive Literature Review) ร่วมกับการวิเคราะห์เชิงเปรียบเทียบ โดยรวบรวมข้อมูลจากแหล่งข้อมูลทางวิชาการที่เชื่อถือได้ ประกอบด้วย

\textbf{แหล่งข้อมูลหลัก} วารสารวิชาการระดับนานาชาติที่ผ่านการประเมินโดยผู้ทรงคุณวุฒิ (Peer-reviewed Journals) เช่น IEEE Transactions on Robotics, International Journal of Robotics Research, และ Journal of Intelligent \& Robotic Systems

\textbf{แหล่งข้อมูลทุติยภูมิ} รายงานการประชุมวิชาการนานาชาติ (Conference Proceedings) มาตรฐานสากล และรายงานวิจัยจากสถาบันชั้นนำ เช่น NIST และ ISO

\textbf{เกณฑ์การคัดเลือกข้อมูล} เน้นงานวิจัยที่ตีพิมพ์ในช่วงปี ค.ศ. 2015-2025 มีการอ้างอิงและความน่าเชื่อถือสูง และเกี่ยวข้องโดยตรงกับหุ่นยนต์อเนกประสงค์หรือแนวคิดที่เกี่ยวข้อง

การวิเคราะห์ข้อมูลใช้การสังเคราะห์เชิงพรรณนา (Narrative Synthesis) โดยจัดกลุ่มข้อมูลตามประเด็นหลัก วิเคราะห์แนวโน้มและความสัมพันธ์ และสรุปเป็นองค์ความรู้ที่เป็นระบบ

\subsection{ประโยชน์ที่คาดว่าจะได้รับ}

การศึกษานี้คาดว่าจะให้ประโยชน์แก่ผู้เกี่ยวข้องหลายกลุ่ม ดังนี้

\textbf{สำหรับนักวิจัยและนักวิชาการ} เป็นการสังเคราะห์องค์ความรู้ที่เป็นปัจจุบันและครอบคลุม สามารถใช้เป็นฐานข้อมูลสำหรับการวิจัยต่อยอดในอนาคต

\textbf{สำหรับผู้ประกอบการและวิศวกร} ให้ข้อมูลสำคัญสำหรับการตัดสินใจลงทุนและการประยุกต์ใช้เทคโนโลยีหุ่นยนต์อเนกประสงค์ในภาคอุตสาหกรรม

\textbf{สำหรับนักศึกษาและผู้ที่สนใจ} เป็นแหล่งข้อมูลการเรียนรู้ที่เป็นระบบเกี่ยวกับเทคโนโลยีหุ่นยนต์ขั้นสูงและแนวโน้มการพัฒนาในอนาคต

\textbf{สำหรับหน่วยงานกำกับดูแล} ให้ข้อมูลประกอบการพิจารณาจัดทำนโยบายและมาตรฐานที่เกี่ยวข้องกับการใช้งานหุ่นยนต์อเนกประสงค์อย่างปลอดภัยและมีประสิทธิภาพ

\section{เนื้อหา}
เนื้อหาของรายงานในส่วนนี้จะครอบคลุมองค์ความรู้หลักเกี่ยวกับหุ่นยนต์อเนกประสงค์ในด้านต่างๆ ที่สำคัญทางวิศวกรรม โดยแบ่งออกเป็น 4 หัวข้อย่อยหลัก ได้แก่ นิยามและแนวคิด สถาปัตยกรรมและการออกแบบ เทคโนโลยีหลักและการควบคุม และการประยุกต์ใช้ในภาคอุตสาหกรรม

\subsection{นิยามและแนวคิดหุ่นยนต์อเนกประสงค์}

หุ่นยนต์อเนกประสงค์ (Polyfunctional Robots) เป็นแนวคิดที่พัฒนาขึ้นเพื่อตอบสนองความต้องการในการใช้งานหุ่นยนต์ที่มีความยืดหยุ่นและสามารถปรับเปลี่ยนฟังก์ชันการทำงานได้ตามสถานการณ์ที่แตกต่างกัน ซึ่งแตกต่างจากหุ่นยนต์แบบดั้งเดิมที่มักออกแบบมาเพื่อปฏิบัติงานเฉพาะอย่างเดียว \parencite{liang2025decoding} การพัฒนาหุ่นยนต์อเนกประสงค์มีพื้นฐานมาจากหลักการของหุ่นยนต์โมดูลาร์และหุ่นยนต์ที่สามารถปรับโครงสร้างได้ด้วยตนเอง แต่มีจุดเน้นที่การเพิ่มความสามารถในการปฏิบัติงานหลากหลายมากกว่าการเปลี่ยนแปลงโครงสร้างเพียงอย่างเดียว

แนวคิดหลักของหุ่นยนต์อเนกประสงค์สามารถแบ่งออกเป็น 3 ประเภทตามกลไกการปรับเปลี่ยนฟังก์ชัน ประการแรก \textbf{การปรับโครงสร้างทางกายภาพ (Physical Reconfiguration)} ซึ่งเป็นการเปลี่ยนแปลงโครงสร้างหรือรูปร่างของหุ่นยนต์เพื่อให้เหมาะสมกับงานใหม่ เช่น การเชื่อมต่อหรือแยกโมดูลต่างๆ การปรับขนาดหรือรูปร่างของส่วนประกอบ \parencite{post2023modular} ประการที่สอง \textbf{การปรับซอฟต์แวร์และอัลกอริทึม (Software Reconfiguration)} เป็นการเปลี่ยนแปลงโปรแกรมควบคุม อัลกอริทึมการเรียนรู้ หรือแผนการทำงานของหุ่นยนต์โดยไม่ต้องเปลี่ยนฮาร์ดแวร์ และประการสุดท้าย \textbf{การผสมผสานโมดูลฮาร์ดแวร์ (Hardware Module Integration)} คือการรวมหรือแยกโมดูลฮาร์ดแวร์ที่มีฟังก์ชันเฉพาะเพื่อสร้างความสามารถใหม่

ความแตกต่างสำคัญระหว่างหุ่นยนต์อเนกประสงค์กับหุ่นยนต์ประเภทอื่นสามารถสรุปได้ดังนี้ \textbf{หุ่นยนต์ทั่วไป (Conventional Robots)} ออกแบบมาเพื่อปฏิบัติงานเฉพาะอย่างเดียวด้วยประสิทธิภาพสูง เช่น หุ่นยนต์เชื่อมโลหะหรือหุ่นยนต์วาดภาพ \textbf{หุ่นยนต์โมดูลาร์ (Modular Robots)} เน้นการประกอบและแยกชิ้นส่วนเพื่อสร้างรูปร่างใหม่ \parencite{bi2016survey} \textbf{หุ่นยนต์ปรับโครงสร้างได้ (Self-Reconfiguring Robots)} สามารถเปลี่ยนรูปร่างโดยอัตโนมัติเพื่อเคลื่อนที่หรือปฏิบัติงานในสภาพแวดล้อมที่แตกต่างกัน \parencite{hameed2017modular} และ \textbf{หุ่นยนต์อเนกประสงค์ (Polyfunctional Robots)} เน้นความสามารถในการปฏิบัติงานหลากหลายประเภทภายในระบบเดียวผ่านการปรับเปลี่ยนทั้งฮาร์ดแวร์และซอฟต์แวร์

การจำแนกประเภทของหุ่นยนต์อเนกประสงค์สามารถแบ่งตามมิติต่างๆ ได้ดังนี้ \textbf{ตามลักษณะการเคลื่อนที่} แบ่งเป็น หุ่นยนต์ติดตั้งคงที่ (Stationary) หุ่นยนต์เคลื่อนที่ได้ (Mobile) และหุ่นยนต์ไฮบริด (Hybrid) \textbf{ตามระดับความอัตโนมัติ} แบ่งเป็น หุ่นยนต์ควบคุมด้วยมือ (Manual) กึ่งอัตโนมัติ (Semi-autonomous) และอัตโนมัติเต็มรูปแบบ (Fully Autonomous) \textbf{ตามขอบเขตการประยุกต์ใช้} แบ่งเป็น หุ่นยนต์อุตสาหกรรม (Industrial) หุ่นยนต์บริการ (Service) และหุ่นยนต์เฉพาะทาง (Specialized) และ \textbf{ตามกลไกการปรับเปลี่ยนฟังก์ชัน} แบ่งเป็น แบบปรับฮาร์ดแวร์ (Hardware-based) แบบปรับซอฟต์แวร์ (Software-based) และแบบผสมผสาน (Hybrid Approach)

\subsection{สถาปัตยกรรมและการออกแบบ}

สถาปัตยกรรมของหุ่นยนต์อเนกประสงค์มีลักษณะเฉพาะที่แตกต่างจากหุ่นยนต์ทั่วไป โดยเน้นการออกแบบแบบโมดูลาร์ที่สามารถปรับโครงสร้างได้ (Modular Reconfigurable Architecture) เพื่อรองรับการเปลี่ยนแปลงฟังก์ชันการทำงานได้อย่างมีประสิทธิภาพ \parencite{liang2025decoding} สถาปัตยกรรมนี้ประกอบด้วยองค์ประกอบหลัก 4 ส่วน ได้แก่

\textbf{โมดูลฮาร์ดแวร์พื้นฐาน (Core Hardware Modules)} เป็นโมดูลหลักที่ประกอบด้วยหน่วยประมวลผลกลาง เซ็นเซอร์พื้นฐาน และระบบสื่อสาร โมดูลนี้ทำหน้าที่เป็นศูนย์กลางการควบคุมและประสานงานกับโมดูลอื่นๆ โดยมีการออกแบบให้มีความทนทานและเสถียรภาพสูง เนื่องจากเป็นส่วนที่ใช้งานตลอดเวลาไม่ว่าหุ่นยนต์จะปฏิบัติงานประเภทใด

\textbf{โมดูลฟังก์ชันเฉพาะ (Specialized Function Modules)} เป็นโมดูลที่ออกแบบมาเพื่อปฏิบัติงานเฉพาะทาง เช่น โมดูลสำหรับการจับยึด (Gripping Module) โมดูลสำหรับการตัด (Cutting Module) โมดูลสำหรับการเชื่อมต่อ (Welding Module) หรือโมดูลสำหรับการวิเคราะห์ (Analysis Module) โมดูลเหล่านี้สามารถเปลี่ยนแปลงหรือเพิ่มเติมได้ตามความต้องการของงาน \parencite{post2023modular}

\textbf{ระบบเชื่อมต่อและอินเทอร์เฟซ (Connection and Interface Systems)} เป็นระบบที่ทำให้โมดูลต่างๆ สามารถเชื่อมต่อและสื่อสารกันได้อย่างราบรื่น ระบบนี้ประกอบด้วยการเชื่อมต่อทางกายภาพ (Physical Connection) การเชื่อมต่อทางไฟฟ้า (Electrical Connection) และการเชื่อมต่อทางข้อมูล (Data Connection) การออกแบบระบบเชื่อมต่อจะต้องคำนึงถึงความแข็งแรง ความเสถียร และความสะดวกในการถอดเปลี่ยน

\textbf{ระบบควบคุมแบบลำดับชั้น (Hierarchical Control System)} เป็นระบบที่จัดการควบคุมการทำงานของหุ่นยนต์ในระดับต่างๆ โดยแบ่งเป็น 3 ระดับหลัก ระดับบน (High-level Control) ทำหน้าที่วางแผนงานโดยรวมและตัดสินใจเชิงกลยุทธ์ ระดับกลาง (Mid-level Control) ดำเนินการแปลงแผนงานเป็นคำสั่งเฉพาะ และระดับล่าง (Low-level Control) ควบคุมการทำงานของแต่ละโมดูลในรายละเอียด \parencite{tassi2024multimodal}

การออกแบบสถาปัตยกรรมแบบโมดูลาร์มีหลักการสำคัญหลายประการ \textbf{หลักการมาตรฐานการเชื่อมต่อ (Standardized Interface)} กำหนดให้โมดูลทุกตัวมีรูปแบบการเชื่อมต่อที่เป็นมาตรฐานเดียวกัน ทำให้สามารถผสมผสานโมดูลต่างประเภทได้อย่างอิสระ \textbf{หลักการแยกฟังก์ชัน (Functional Separation)} แยกฟังก์ชันต่างๆ ออกเป็นโมดูลอิสระ ทำให้สามารถพัฒนา ปรับปรุง หรือเปลี่ยนแปลงแต่ละฟังก์ชันได้โดยไม่กระทบต่อส่วนอื่น \textbf{หลักการความยืดหยุ่น (Flexibility Principle)} ออกแบบให้ระบบสามารถรองรับการเพิ่มหรือลดโมดูลได้ตามความต้องการ และ \textbf{หลักการความปลอดภัย (Safety Principle)} มีระบบป้องกันการทำงานผิดพลาดเมื่อมีการเปลี่ยนแปลงโครงสร้าง

การออกแบบเชิงกลไกมีความสำคัญอย่างยิ่งต่อความสำเร็จของหุ่นยนต์อเนกประสงค์ โดยเฉพาะการออกแบบกลไกการเชื่อมต่อ (Coupling Mechanisms) ที่ต้องคำนึงถึงปัจจัยหลายประการ ได้แก่ ความแข็งแรงเชิงกล (Mechanical Strength) ต้องรองรับน้ำหนักและแรงกระทำจากการใช้งาน ความง่ายในการเชื่อมต่อและถอด (Ease of Connection/Disconnection) เพื่อให้สามารถปรับเปลี่ยนโครงสร้างได้รวดเร็ว ความแม่นยำในการวางตำแหน่ง (Positioning Accuracy) เพื่อให้การเชื่อมต่อมีความแม่นยำสูง และความทนทานต่อสภาพแวดล้อม (Environmental Durability) เพื่อใช้งานได้ในสภาพแวดล้อมที่หลากหลาย \parencite{krishnan2023deep}

\subsection{เทคโนโลยีหลักและการควบคุม}

เทคโนโลยีหลักที่ขับเคลื่อนหุ่นยนต์อเนกประสงค์ประกอบด้วยองค์ประกอบสำคัญหลายด้าน โดยเฉพาะการบูรณาการเซ็นเซอร์แบบหลายโหมด (Multimodal Sensor Integration) ซึ่งเป็นเทคโนโลยีที่ทำให้หุ่นยนต์สามารถรับรู้และตีความข้อมูลจากสภาพแวดล้อมได้อย่างครอบคลุมและแม่นยำ \parencite{yang2024body} ระบบเซ็นเซอร์แบบหลายโหมดผสมผสานเซ็นเซอร์ประเภทต่างๆ เข้าด้วยกัน เช่น เซ็นเซอร์ภาพ (Vision Sensors) เซ็นเซอร์การสัมผัส (Tactile Sensors) เซ็นเซอร์เสียง (Audio Sensors) เซ็นเซอร์แรง (Force Sensors) และเซ็นเซอร์อุณหภูมิ (Temperature Sensors) เพื่อสร้างการรับรู้ที่ครอบคลุมและสามารถปรับใช้กับงานที่หลากหลาย

ระบบเซ็นเซอร์ที่มีการกระจายตัวบนผิวหน้าของหุ่นยนต์ (Distributed Surface Sensing) เป็นนวัตกรรมสำคัญที่ช่วยให้หุ่นยนต์อเนกประสงค์สามารถรับรู้การสัมผัสและปฏิสัมพันธ์กับวัตถุได้ทั่วทั้งร่างกาย การพัฒนาเซ็นเซอร์แบบนี้ใช้เทคโนโลยีผิวหนังเทียม (Artificial Skin) ที่มีความยืดหยุ่นและสามารถติดตั้งบนโครงสร้างที่มีรูปร่างโค้งหรือซับซ้อน ทำให้หุ่นยนต์สามารถทำงานที่ต้องการความละเอียดอ่อนในการจับต้องและจัดการวัตถุได้อย่างมีประสิทธิภาพ

แอคชูเอเตอร์ปรับความแข็งได้ (Variable Stiffness Actuators) เป็นเทคโนโลยีอีกหนึ่งเทคโนโลยีหลักที่ทำให้หุ่นยนต์อเนกประสงค์สามารถปรับเปลี่ยนลักษณะการเคลื่อนไหวและการตอบสนองต่อแรงภายนอกได้ตามลักษณะของงาน \parencite{ieee2024compact} แอคชูเอเตอร์ประเภทนี้สามารถปรับเปลี่ยนระดับความแข็งแรงและความยืดหยุ่นได้ในระหว่างการทำงาน ทำให้หุ่นยนต์สามารถสลับระหว่างการทำงานที่ต้องการความแม่นยำสูง (เช่น การประกอบชิ้นส่วนที่ต้องการความแข็งแรง) และการทำงานที่ต้องการความนุ่มนวล (เช่น การจัดการวัตถุที่บอบบาง) ได้อย่างมีประสิทธิภาพ

ระบบควบคุมแบบลำดับชั้นเป็นแกนหลักในการจัดการความซับซ้อนของหุ่นยนต์อเนกประสงค์ ระบบนี้แบ่งการควบคุมออกเป็นหลายระดับ โดยแต่ละระดับมีหน้าที่และขอบเขตความรับผิดชอบที่ชัดเจน \textbf{ระดับควบคุมสูง (High-level Control)} ทำหน้าที่วางแผนภารกิจโดยรวม (Mission Planning) การตัดสินใจเชิงกลยุทธ์ (Strategic Decision Making) และการจัดการทรัพยากร (Resource Management) ระดับนี้ใช้ปัญญาประดิษฐ์และอัลกอริทึมการเรียนรู้เพื่อวิเคราะห์สถานการณ์และกำหนดแผนการทำงาน \textbf{ระดับควบคุมกลาง (Mid-level Control)} ทำหน้าที่แปลงแผนการทำงานจากระดับสูงเป็นคำสั่งเฉพาะ (Task Decomposition) การประสานงานระหว่างโมดูล (Inter-module Coordination) และการตรวจสอบความปลอดภัย (Safety Monitoring) และ \textbf{ระดับควบคุมต่ำ (Low-level Control)} ทำหน้าที่ควบคุมการทำงานของแต่ละแอคชูเอเตอร์และเซ็นเซอร์ในรายละเอียด (Actuator Control และ Sensor Data Processing) \parencite{tassi2024multimodal}

การประยุกต์ใช้ปัญญาประดิษฐ์แบบโมเดลพื้นฐาน (Foundation Models) ในการควบคุมหุ่นยนต์อเนกประสงค์เป็นแนวทางใหม่ที่มีศักยภาพสูงในการเพิ่มความสามารถในการปรับตัวและเรียนรู้งานใหม่ โมเดลพื้นฐานเหล่านี้ได้รับการฝึกฝนด้วยข้อมูลจำนวนมหาศาลและสามารถนำมาปรับใช้กับงานใหม่ได้โดยต้องการข้อมูลฝึกเพิ่มเติมเพียงเล็กน้อย การใช้โมเดลพื้นฐานช่วยให้หุ่นยนต์สามารถเรียนรู้และปรับตัวเพื่อทำงานใหม่ได้เร็วขึ้น รวมถึงสามารถโอนถ่ายความรู้จากงานหนึ่งไปยังอีกงานหนึ่งได้อย่างมีประสิทธิภาพ

เทคโนโลยีการเรียนรู้แบบต่อเนื่อง (Continual Learning) เป็นองค์ประกอบสำคัญที่ช่วยให้หุ่นยนต์อเนกประสงค์สามารถเรียนรู้และปรับปรุงประสิทธิภาพการทำงานได้ตลอดเวลา โดยไม่สูญเสียความรู้เดิมที่ได้เรียนรู้มาแล้ว เทคโนโลยีนี้ใช้อัลกอริทึมพิเศษที่ป้องกันปัญหา Catastrophic Forgetting และช่วยให้หุ่นยนต์สามารถสะสมความรู้และประสบการณ์ไปเรื่อยๆ จากการทำงานหลากหลายประเภท

\subsection{การประยุกต์ใช้ในภาคอุตสาหกรรม}

การประยุกต์ใช้หุ่นยนต์อเนกประสงค์ในภาคอุตสาหกรรมได้รับความสนใจเพิ่มขึ้นอย่างมากในยุคของการปฏิวัติอุตสาหกรรม 4.0 โดยเฉพาะในการผลิตอัตโนมัติที่ต้องการความยืดหยุ่นในการปรับเปลี่ยนประเภทผลิตภัณฑ์หรือกระบวนการผลิต \parencite{mohammadi2023mobile} หุ่นยนต์อเนกประสงค์ประเภทโมบายแมนิปิวเลเตอร์ (Mobile Manipulators) ได้รับการพัฒนาและนำไปใช้งานในโรงงานผลิตรถยนต์ โรงงานอิเล็กทรอนิกส์ และโรงงานอาหาร เนื่องจากสามารถเคลื่อนที่ไปยังจุดต่างๆ ในสายการผลิตและปฏิบัติงานหลากหลาย เช่น การหยิบจับ การประกอบ การตรวจสอบคุณภาพ และการบรรจุหีบห่อในระบบเดียว

ในอุตสาหกรรมการผลิตชิ้นส่วนอิเล็กทรอนิกส์ หุ่นยนต์อเนกประสงค์ถูกใช้ในกระบวนการประกอบแผงวงจรพิมพ์ (PCB Assembly) ซึ่งต้องการการปรับเปลี่ยนเครื่องมือและวิธีการทำงานบ่อยครั้งตามประเภทของแผงวงจรที่แตกต่างกัน หุ่นยนต์เหล่านี้สามารถเปลี่ยนจากการติดตั้งชิ้นส่วนขนาดใหญ่เป็นชิ้นส่วนขนาดเล็กที่ต้องการความแม่นยำสูง โดยการปรับเปลี่ยนเครื่องมือปลายมือ (End-effector) และพารามิเตอร์การควบคุมแบบอัตโนมัติ

การประยุกต์ใช้ในภาคการแพทย์และสุขภาพเป็นอีกหนึ่งด้านที่มีความก้าวหน้าอย่างรวดเร็ว หุ่นยนต์อเนกประสงค์ทางการแพทย์สามารถปฏิบัติงานหลากหลาย ตั้งแต่การช่วยเหลือในการผ่าตัด การฟื้นฟูสมรรถภาพ การดูแลผู้ป่วย ไปจนถึงการขนส่งยาและอุปกรณ์การแพทย์ภายในโรงพยาบาล ระบบหุ่นยนต์เหล่านี้ต้องมีความปลอดภัยสูงและสามารถปรับเปลี่ยนระหว่างการทำงานแบบเข้มงวดที่ต้องการความแม่นยำสูง (เช่น การผ่าตัด) และการทำงานแบบอ่อนโยนที่ต้องการปฏิสัมพันธ์กับมนุษย์ (เช่น การดูแลผู้ป่วย) \parencite{stueckler2023hollie}

ในภาคการสำรวจอวกาศและการกู้ภัยพิบัติ หุ่นยนต์อเนกประสงค์มีบทบาทสำคัญเนื่องจากสภาพแวดล้อมที่ไม่แน่นอนและการเปลี่ยนแปลงของภารกิจอย่างรวดเร็ว หุ่นยนต์สำรวจอวกาศสามารถปรับเปลี่ยนจากการขุดเจาะดินเพื่อเก็บตัวอย่างเป็นการติดตั้งอุปกรณ์วิทยาศาสตร์หรือการซ่อมแซมยานอวกาศได้ตามความจำเป็น ส่วนหุ่นยนต์กู้ภัยสามารถปฏิบัติงานหลากหลาย เช่น การค้นหาและการช่วยเหลือผู้ประสบภัย การขจัดสิ่งกีดขวาง การนำส่งเสบียงและยา และการประเมินความเสียหายในพื้นที่ที่มนุษย์เข้าไปไม่ได้

ในภาคการเกษตรสมัยใหม่ (Precision Agriculture) หุ่นยนต์อเนกประสงค์ได้รับการพัฒนาให้สามารถปฏิบัติงานตลอดวงจรการเกษตร ตั้งแต่การเตรียมดิน การปลูก การดูแลรักษา การเก็บเกี่ยว และการคัดแยกผลผลิต หุ่นยนต์เหล่านี้ใช้เทคโนโลยีการมองเห็นด้วยคอมพิวเตอร์ (Computer Vision) ร่วมกับเซ็นเซอร์ต่างๆ เพื่อระบุสภาพของพืช วัชพืช และโรคพืช จากนั้นจึงปรับเปลี่ยนเครื่องมือและวิธีการทำงานให้เหมาะสมกับแต่ละสถานการณ์ เช่น การเปลี่ยนจากการพ่นปุ่ยเป็นการกำจัดวัชพืชหรือการเก็บเกี่ยวผลไม้

การประยุกต์ใช้ในระบบโลจิสติกส์และคลังสินค้าแสดงให้เห็นถึงศักยภาพของหุ่นยนต์อเนกประสงค์ในการปรับตัวต่อสภาพแวดล้อมที่เปลี่ยนแปลงตลอดเวลา หุ่นยนต์ในคลังสินค้าสามารถปฏิบัติงานหลากหลาย เช่น การรับสินค้า การจัดเก็บ การหยิบสินค้า การบรรจุหีบห่อ และการขนส่งภายในคลัง โดยสามารถปรับเปลี่ยนระหว่างการจัดการสินค้าที่มีขนาดน้ำหนักและรูปร่างแตกต่างกันได้อย่างอัตโนมัติ \parencite{xie2024pose}

การบูรณาการกับระบบโคบอท (Collaborative Robots) เป็นแนวทางที่ได้รับความนิยมเพิ่มขึ้นในสภาพแวดล้อมอุตสาหกรรมที่ต้องการทั้งความยืดหยุ่นจากหุ่นยนต์และความคิดสร้างสรรค์จากมนุษย์ \parencite{vitolo2022mobile} การผสมผสานระหว่างโมบายแมนิปิวเลเตอร์กับโคบอทช่วยให้สามารถสร้างระบบการผลิตที่มีความยืดหยุ่นสูง โดยโคบอททำงานร่วมกับมนุษย์ในงานที่ต้องการความละเอียดอ่อนและการตัดสินใจ ขณะที่โมบายแมนิปิวเลเตอร์ดำเนินการงานที่ต้องการการเคลื่อนที่และการปรับเปลี่ยนบ่อยครั้ง

ในด้านการประเมินประสิทธิภาพและการวัดผล หน่วยงาน NIST (National Institute of Standards and Technology) ได้พัฒนามาตรฐานการวัดประสิทธิภาพของโมบายแมนิปิวเลเตอร์ในงานประกอบเพื่อการผลิต \parencite{bostelman2016mobile} การวัดผลครอบคลุมด้านต่างๆ เช่น ความแม่นยำในการวางตำแหน่ง (Positioning Accuracy) ความเร็วในการปฏิบัติงาน (Task Execution Speed) ความสามารถในการปรับตัว (Adaptability) และความปลอดภัยในการทำงาน (Safety Performance) การมีมาตรฐานเหล่านี้ช่วยให้ผู้ใช้งานสามารถประเมินและเปรียบเทียบประสิทธิภาพของหุ่นยนต์อเนกประสงค์ต่างๆ ได้อย่างเป็นระบบ

ความท้าทายในการนำหุ่นยนต์อเนกประสงค์ไปประยุกต์ใช้ในอุตสาหกรรมรวมถึงประเด็นหลายด้าน ประการแรก \textbf{ความซับซ้อนในการเขียนโปรแกรม} เนื่องจากต้องการการจัดการหลายฟังก์ชันและการปรับเปลี่ยนระหว่างงานต่างๆ ประการที่สอง \textbf{ต้นทุนการลงทุนสูง} เนื่องจากต้องการเทคโนโลยีขั้นสูงและระบบควบคุมที่ซับซ้อน ประการที่สามเสมอ \textbf{ความต้องการการฝึกอบรมบุคลากร} เพื่อให้สามารถใช้งานและบำรุงรักษาระบบได้อย่างมีประสิทธิภาพ และ \textbf{ประเด็นด้านความปลอดภัย} โดยเฉพาะในสภาพแวดล้อมที่มีการทำงานร่วมกันระหว่างมนุษย์และหุ่นยนต์

แนวโน้มการพัฒนาในอนาคตของการประยุกต์ใช้หุ่นยนต์อเนกประสงค์ในอุตสาหกรรมมุ่งเน้นไปที่การเพิ่มระดับปัญญาประดิษฐ์และความสามารถในการเรียนรู้อัตโนมัติ การพัฒนาระบบการรับรู้ที่ซับซ้อนขึ้น การปรับปรุงความปลอดภัยและความเชื่อถือได้ รวมถึงการลดต้นทุนเพื่อให้สามารถเข้าถึงได้ง่ายขึ้นสำหรับธุรกิจขนาดกลางและขนาดเล็ก การพัฒนาเหล่านี้คาดว่าจะทำให้หุ่นยนต์อเนกประสงค์กลายเป็นองค์ประกอบหลักของระบบการผลิตที่ชาญฉลาดและยืดหยุ่นในอนาคต

\section{การวิเคราะห์และอภิปราย}
% Section 3 content (การวิเคราะห์และอภิปราย)

\section{เอกสารประมิณขายงานฉบับสมบูรณ์}
% Section 4 content (เอกสารประมิณ - Assessment criteria)

\section{ความคิดสร้างสรรค์และความเยียบร้อย}
% Section 5 content (ความคิดสร้างสรรค์)

\section{สรุป}
% Section 6 content (สรุป)

% ----- Bibliography -----
\printbibliography[title=เอกสารอ้างอิง]

% ----- Appendices -----
\appendix

\section{ภาคผนวก ก: คำศัพท์เทคนิค}
% Technical glossary

\section{ภาคผนวก ข: ตารางเปรียบเทียบเทคโนโลยี}
% Technology comparison tables

\end{document}