% !TEX TS-program = xelatex
\documentclass[a4paper]{article}
\usepackage[left=1in, right=1in, top=1in, bottom=1in]{geometry}
\usepackage{fontspec}
\usepackage{polyglossia}
\setdefaultlanguage{thai}
\setotherlanguage{english}
\usepackage{graphicx}

% Font setup for Thai with proper line breaking
\newfontfamily\thaifont{TH Sarabun New}[
  Script=Thai,
  Scale=1.0,
  WordSpace=1.0,
  LetterSpace=0.0
]
\newfontfamily\englishfont{TH Sarabun New}[
  Script=Latin,
  Scale=1.0
]
\setmainfont{TH Sarabun New}

% Enable Thai line breaking and hyphenation
\XeTeXlinebreaklocale "th"
\XeTeXlinebreakskip = 0pt plus 1pt minus 1pt

% Single spacing with proper line height for Thai
\usepackage{setspace}
\setstretch{1.2}

% 16pt body size with better line height
\makeatletter
\renewcommand\normalsize{%
  \@setfontsize\normalsize{16pt}{19.2pt}%
}
\normalsize
\makeatother

% Section titles formatting
\usepackage{titlesec}
\titleformat{\section}{\fontsize{18pt}{21.6pt}\selectfont\bfseries}{\thesection}{1em}{}
\titleformat{\subsection}{\fontsize{16pt}{19.2pt}\selectfont\bfseries}{\thesubsection}{1em}{}
\titleformat{\subsubsection}{\fontsize{16pt}{19.2pt}\selectfont\bfseries}{\thesubsubsection}{1em}{}

% TOC formatting
\usepackage{tocloft}
\renewcommand{\cftsecfont}{\fontsize{16pt}{19.2pt}\selectfont}
\renewcommand{\cftsubsecfont}{\fontsize{16pt}{19.2pt}\selectfont}
\renewcommand{\cftsecpagefont}{\fontsize{16pt}{19.2pt}\selectfont}
\renewcommand{\cftsubsecpagefont}{\fontsize{16pt}{19.2pt}\selectfont}

% Essential packages
\usepackage[hidelinks]{hyperref}
\usepackage{booktabs}
\usepackage{csquotes}

% Thai text formatting improvements for formal documents
\usepackage{ragged2e}
\justifying % Full justification for formal thesis
\hyphenpenalty=10000 % Disable hyphenation for Thai
\exhyphenpenalty=10000

% Bibliography setup for APA
\usepackage[style=apa,backend=biber]{biblatex}
\DeclareLanguageMapping{english}{english-apa}
\addbibresource{references.bib}

% Thai alphabet page numbering - complete implementation
\makeatletter
\def\@thaialph#1{%
  \ifcase#1\or ก\or ข\or ค\or ง\or จ\or ฉ\or ช\or ซ\or ฌ\or ญ\or
  ฎ\or ฏ\or ฐ\or ฑ\or ฒ\or ณ\or ด\or ต\or ถ\or ท\or ธ\or น\or บ\or ป\or ผ\or ฝ\or พ\or ฟ\or ภ\or ม\or
  ย\or ร\or ล\or ว\or ศ\or ษ\or ส\or ห\or ฬ\or อ\or ฮ\else\@ctrerr\fi
}
\def\thaialph#1{\expandafter\@thaialph\csname c@#1\endcsname}
\makeatother

% Formal Thai thesis paragraph settings
\setlength{\parindent}{2em}   % Standard Thai thesis indentation
\setlength{\parskip}{0pt}     % No space between paragraphs

% Better text justification for Thai
\tolerance=1000
\pretolerance=800
\emergencystretch=3em

% Cover page title setup
\date{}
\author{}
\title{
    \includegraphics[width=1in, height=1in, keepaspectratio]{logo.png}
    \\[2ex]
    {\fontsize{32pt}{36pt}\selectfont\textbf{Polyfunctional Robots}}
}

\begin{document}

% ----- Cover Page (no page number) -----
\maketitle
\thispagestyle{empty}

\vfill
\begin{center}
    {\fontsize{22pt}{26pt}\selectfont\textbf{
        นายคมชาญ วิเศษนคร
        \\ 663040419-1
    }}
\end{center}
\vfill

\vfill
\begin{center}
    {\fontsize{16pt}{20pt}\selectfont\textbf{
        รายงานฉบับนี้เป็นส่วนหนึ่งของรายวิชา EN813761 การสัมมนาทางวิศวกรรมคอมพิวเตอร์
        \\ สาขาวิชาวิศวกรรมคอมพิวเตอร์ มหาวิทยาลัยขอนแก่น
        \\ ภาคเรียนที่ 1 ปีการศึกษา 2568
    }}
\end{center}

\newpage

% ----- Start Thai letter page numbering for front matter -----
\pagenumbering{thaialph}
\setcounter{page}{1}

% ----- Abstract (Thai) -----
{\centering
    {\fontsize{18pt}{21.6pt}\selectfont\textbf{บทคัดย่อ}\par}
}
\vspace{1em}

% Thai abstract content goes here

\vspace{1em}

% ----- Abstract (English) -----
{\centering
    {\fontsize{18pt}{21.6pt}\selectfont\textbf{Abstract}\par}
}
\vspace{1em}

% English abstract content goes here

\vspace{1em}

\noindent
\textbf{คำสำคัญ:} หุ่นยนต์อเนกประสงค์, หุ่นยนต์โมดูลาร์, การควบคุมแบบลำดับชั้น, เซ็นเซอร์หลายโหมด, แอคชูเอเตอร์ปรับความแข็งได้

\vspace{0.5em}

\noindent
\textbf{Keywords:} Polyfunctional Robots, Modular Robotics, Hierarchical Control, Multimodal Sensors, Variable Stiffness Actuators

\newpage

% ----- Table of Contents -----
\tableofcontents
\newpage

% ----- Switch to Arabic numbering for main content -----
\pagenumbering{arabic}
\setcounter{page}{1}

% ----- Unnumbered front matter sections -----
\section*{คำนำ}
% คำนำ content goes here

\newpage

\section*{องค์ประกอบของรายงาน}
% รายละเอียดองค์ประกอบตามเกณฑ์การประเมิน

\newpage

% ----- Main Content Sections (Numbered) -----
\section{บทนำ} 
% KEEP CURRENT CONTENT - Already excellent
% 1.1 ความเป็นมาและความสำคัญของปัญหา
% 1.2 วัตถุประสงค์ของการศึกษา  
% 1.3 ขอบเขตการศึกษา
% 1.4 คำจำกัดความสำคัญ
% 1.5 ประโยชน์ที่คาดว่าจะได้รับ
% 1.6 โครงสร้างของรายงาน [ADD NEW]

\section{การทบทวนวรรณกรรมที่เกี่ยวข้อง} 
% RENAME from "เนื้อหา" - EXPAND CURRENT CONTENT
% 2.1 แนวคิดและนิยามหุ่นยนต์อเนกประสงค์
% 2.2 สถาปัตยกรรมและการออกแบบระบบ
% 2.3 เทคโนโลยีหลักและระบบควบคุม  
% 2.4 การประยุกต์ใช้ในภาคอุตสาหกรรม
% 2.5 ช่องว่างความรู้และแนวทางการวิจัย [ADD NEW]

\section{ระเบียบวิธีวิจัย}
% EXTRACT from Section 1 + EXPAND
% 3.1 แนวทางการศึกษา
% 3.2 การรวบรวมข้อมูล  
% 3.3 เกณฑ์การคัดเลือกแหล่งข้อมูล
% 3.4 การวิเคราะห์และสังเคราะห์ข้อมูล
% 3.5 ข้อจำกัดของการศึกษา

\section{ผลการศึกษาและการวิเคราะห์}
% NEW SYNTHESIS CHAPTER - Critical analysis
% 4.1 การวิเคราะห์เชิงเปรียบเทียบเทคโนโลยี
% 4.2 การประเมินประสิทธิภาพและข้อจำกัด
% 4.3 แนวโน้มการพัฒนาและทิศทางอนาคต  
% 4.4 ความท้าทายและปัจจัยสำคัญ
% 4.5 กรอบแนวคิดเชิงบูรณาการ [SYNTHESIS]

\section{การอภิปรายผล}
% CRITICAL DISCUSSION - Higher-order analysis
% 5.1 การอภิปรายผลการวิเคราะห์
% 5.2 ผลกระทบต่อภาคอุตสาหกรรม
% 5.3 ข้อเสนอแนะเชิงนโยบายและการปฏิบัติ
% 5.4 ข้อจำกัดและข้อควรระวัง

\section{สรุปและข้อเสนอแนะ}
% COMPREHENSIVE CONCLUSIONS
% 6.1 สรุปผลการศึกษา
% 6.2 ข้อเสนอแนะสำหรับการวิจัยต่อไป  
% 6.3 ข้อเสนอแนะสำหรับการนำไปประยุกต์ใช้
% 6.4 คำสุดท้าย

% ----- Bibliography -----
\printbibliography[title=เอกสารอ้างอิง]

% ----- Appendices -----
\appendix

\section{ภาคผนวก ก: คำศัพท์เทคนิค}
% Technical glossary

\section{ภาคผนวก ข: ตารางเปรียบเทียบเทคโนโลยี}
% Technology comparison tables

\end{document}