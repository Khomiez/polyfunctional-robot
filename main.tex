% !TEX TS-program = xelatex
\documentclass[a4paper]{article}
\usepackage[left=1in, right=1in, top=1in, bottom=1in]{geometry}
\usepackage{fontspec}
\usepackage{polyglossia}
\setdefaultlanguage{thai}
\setotherlanguage{english}
\usepackage{graphicx}

% Font setup for Thai with proper line breaking
\newfontfamily\thaifont{TH Sarabun New}[
  Script=Thai,
  Scale=1.0,
  WordSpace=1.0,
  LetterSpace=0.0
]
\newfontfamily\englishfont{TH Sarabun New}[
  Script=Latin,
  Scale=1.0
]
\setmainfont{TH Sarabun New}

% Enable Thai line breaking and hyphenation
\XeTeXlinebreaklocale "th"
\XeTeXlinebreakskip = 0pt plus 1pt minus 1pt

% Single spacing with proper line height for Thai
\usepackage{setspace}
\setstretch{1.2}

% 16pt body size with better line height
\makeatletter
\renewcommand\normalsize{%
  \@setfontsize\normalsize{16pt}{19.2pt}%
}
\normalsize
\makeatother

% Section titles formatting
\usepackage{titlesec}
\titleformat{\section}{\fontsize{18pt}{21.6pt}\selectfont\bfseries}{\thesection}{1em}{}
\titleformat{\subsection}{\fontsize{16pt}{19.2pt}\selectfont\bfseries}{\thesubsection}{1em}{}
\titleformat{\subsubsection}{\fontsize{16pt}{19.2pt}\selectfont\bfseries}{\thesubsubsection}{1em}{}

% TOC formatting
\usepackage{tocloft}
\renewcommand{\cftsecfont}{\fontsize{16pt}{19.2pt}\selectfont}
\renewcommand{\cftsubsecfont}{\fontsize{16pt}{19.2pt}\selectfont}
\renewcommand{\cftsecpagefont}{\fontsize{16pt}{19.2pt}\selectfont}
\renewcommand{\cftsubsecpagefont}{\fontsize{16pt}{19.2pt}\selectfont}

% Essential packages
\usepackage[hidelinks]{hyperref}
\usepackage{booktabs}
\usepackage{csquotes}

% Thai text formatting improvements for formal documents
\usepackage{ragged2e}
\justifying % Full justification for formal thesis
\hyphenpenalty=10000 % Disable hyphenation for Thai
\exhyphenpenalty=10000

% Bibliography setup for APA
\usepackage[style=apa,backend=biber]{biblatex}
\DeclareLanguageMapping{english}{english-apa}
\addbibresource{references.bib}

% Thai alphabet page numbering - complete implementation
\makeatletter
\def\@thaialph#1{%
  \ifcase#1\or ก\or ข\or ค\or ง\or จ\or ฉ\or ช\or ซ\or ฌ\or ญ\or
  ฎ\or ฏ\or ฐ\or ฑ\or ฒ\or ณ\or ด\or ต\or ถ\or ท\or ธ\or น\or บ\or ป\or ผ\or ฝ\or พ\or ฟ\or ภ\or ม\or
  ย\or ร\or ล\or ว\or ศ\or ษ\or ส\or ห\or ฬ\or อ\or ฮ\else\@ctrerr\fi
}
\def\thaialph#1{\expandafter\@thaialph\csname c@#1\endcsname}
\makeatother

% Formal Thai thesis paragraph settings
\setlength{\parindent}{2em}   % Standard Thai thesis indentation
\setlength{\parskip}{0pt}     % No space between paragraphs

% Better text justification for Thai
\tolerance=1000
\pretolerance=800
\emergencystretch=3em

% Cover page title setup
\date{}
\author{}
\title{
    \includegraphics[width=1in, height=1in, keepaspectratio]{logo.png}
    \\[2ex]
    {\fontsize{32pt}{36pt}\selectfont\textbf{Polyfunctional Robots}}
}

\begin{document}

% ----- Cover Page (no page number) -----
\maketitle
\thispagestyle{empty}

\vfill
\begin{center}
    {\fontsize{22pt}{26pt}\selectfont\textbf{
        นายคมชาญ วิเศษนคร
        \\ 663040419-1
    }}
\end{center}
\vfill

\vfill
\begin{center}
    {\fontsize{16pt}{20pt}\selectfont\textbf{
        รายงานฉบับนี้เป็นส่วนหนึ่งของรายวิชา EN813761 การสัมมนาทางวิศวกรรมคอมพิวเตอร์
        \\ สาขาวิชาวิศวกรรมคอมพิวเตอร์ มหาวิทยาลัยขอนแก่น
        \\ ภาคเรียนที่ 1 ปีการศึกษา 2568
    }}
\end{center}

\newpage

% ----- Start Thai letter page numbering for front matter -----
\pagenumbering{thaialph}
\setcounter{page}{1}

% ----- Abstract (Thai) -----
{\centering
    {\fontsize{18pt}{21.6pt}\selectfont\textbf{บทคัดย่อ}\par}
}
\vspace{1em}

% Thai abstract content goes here

\vspace{1em}

% ----- Abstract (English) -----
{\centering
    {\fontsize{18pt}{21.6pt}\selectfont\textbf{Abstract}\par}
}
\vspace{1em}

% English abstract content goes here

\vspace{1em}

\noindent
\textbf{คำสำคัญ:} หุ่นยนต์อเนกประสงค์, หุ่นยนต์โมดูลาร์, การควบคุมแบบลำดับชั้น, เซ็นเซอร์หลายโหมด, แอคชูเอเตอร์ปรับความแข็งได้

\vspace{0.5em}

\noindent
\textbf{Keywords:} Polyfunctional Robots, Modular Robotics, Hierarchical Control, Multimodal Sensors, Variable Stiffness Actuators

\newpage

% ----- Table of Contents -----
\tableofcontents
\newpage

% ----- Switch to Arabic numbering for main content -----
\pagenumbering{arabic}
\setcounter{page}{1}

% ----- Unnumbered front matter sections -----
\section*{คำนำ}
% คำนำ content goes here

\newpage

\section*{องค์ประกอบของรายงาน}
% รายละเอียดองค์ประกอบตามเกณฑ์การประเมิน

\newpage

% ----- Main Content Sections (Numbered) -----

\section{บทนำ}
หุ่นยนต์อเนกประสงค์ (Polyfunctional Robots) หรือหุ่นยนต์อเนกฟังก์ชัน เป็นระบบหุ่นยนต์ขั้นสูงที่ออกแบบให้สามารถปฏิบัติภารกิจที่หลากหลายและซับซ้อนภายในระบบเดียว โดยไม่จำเป็นต้องมีการปรับเปลี่ยนฮาร์ดแวร์หลักอย่างมีนัยสำคัญ \parencite{liang2025decoding} ซึ่งแตกต่างจากหุ่นยนต์แบบดั้งเดิมที่มักออกแบบมาเพื่อปฏิบัติงานเฉพาะทางเพียงอย่างเดียว หุ่นยนต์อเนกประสงค์สามารถปรับเปลี่ยนฟังก์ชันการทำงานผ่านการปรับโครงสร้างทางกายภาพ (Physical Reconfiguration) การเขียนโปรแกรมควบคุมใหม่ (Software Reconfiguration) หรือการผสมผสานโมดูลต่างๆ เข้าด้วยกัน \parencite{post2023modular}

ในยุคของการปฏิวัติอุตสาหกรรม 4.0 และการพัฒนาปัญญาประดิษฐ์ ความต้องการหุ่นยนต์ที่มีความยืดหยุ่นและสามารถปรับตัวได้กับสภาพแวดล้อมที่เปลี่ยนแปลงเพิ่มขึ้นอย่างรวดเร็ว \parencite{mohammadi2023mobile} หุ่นยนต์อเนกประสงค์จึงกลายเป็นทางเลือกที่มีความสำคัญสำหรับอุตสาหกรรมต่างๆ ตั้งแต่การผลิตและการประกอบชิ้นส่วน ไปจนถึงการแพทย์และการสำรวจอวกาศ เนื่องจากสามารถลดต้นทุนการลงทุนและเพิ่มประสิทธิภาพการใช้งานผ่านการใช้ระบบเดียวสำหรับหลายงาน

แนวคิดของหุ่นยนต์อเนกประสงค์มีความเชื่อมโยงอย่างใกล้ชิดกับหุ่นยนต์โมดูลาร์ (Modular Robots) และหุ่นยนต์ที่ปรับโครงสร้างได้ด้วยตนเอง (Self-Reconfiguring Robots) \parencite{seo2019modular} อย่างไรก็ตาม หุ่นยนต์อเนกประสงค์มีจุดเน้นที่แตกต่างออกไป คือ การเน้นที่ความสามารถในการปฏิบัติงานหลากหลายประเภทมากกว่าการเปลี่ยนรูปร่างหรือโครงสร้าง ทำให้เหมาะสำหรับการประยุกต์ใช้ในสภาพแวดล้อมที่ต้องการความเชี่ยวชาญในหลายด้านพร้อมกัน

\subsection{วัตถุประสงค์ของการศึกษา}

การศึกษานี้มีวัตถุประสงค์หลักเพื่อวิเคราะห์และสังเคราะห์องค์ความรู้เกี่ยวกับหุ่นยนต์อเนกประสงค์ในมุมมองทางวิศวกรรม โดยมีจุดมุ่งหมายเฉพาะ ดังนี้

\textbf{1. วิเคราะห์สถาปัตยกรรมและการออกแบบ} เพื่อศึกษาหลักการออกแบบสถาปัตยกรรมแบบโมดูลาร์ที่ปรับโครงสร้างได้ (Modular Reconfigurable Architecture) และระบบควบคุมแบบลำดับชั้น (Hierarchical Control Systems) ที่เป็นพื้นฐานสำคัญของหุ่นยนต์อเนกประสงค์ \parencite{tassi2024multimodal}

\textbf{2. ศึกษาเทคโนโลยีหลัก} โดยเฉพาะการบูรณาการเซ็นเซอร์แบบหลายโหมด (Multimodal Sensor Integration) \parencite{yang2024body} แอคชูเอเตอร์ปรับความแข็งได้ (Variable Stiffness Actuators) และการประยุกต์ใช้ปัญญาประดิษฐ์แบบโมเดลพื้นฐาน (Foundation Models) ในการควบคุม

\textbf{3. วิเคราะห์การประยุกต์ใช้} ในภาคอุตสาหกรรมสำคัญ รวมถึงการผลิตอัตโนมัติ การแพทย์ การสำรวจอวกาศ และการกู้ภัยพิบัติ พร้อมทั้งประเมินประสิทธิภาพและความคุ้มค่าทางเศรษฐศาสตร์

\textbf{4. ระบุความท้าทายและข้อจำกัด} ทั้งในด้านเทคนิคและการนำไปใช้งานจริง รวมถึงประเด็นด้านความปลอดภัยและมาตรฐานสากล เช่น ISO 10218-1:2025 \parencite{iso2025robotics}

\subsection{ขอบเขตการศึกษา}

การศึกษานี้มุ่งเน้นหุ่นยนต์อเนกประสงค์ในบริบททางวิศวกรรม โดยครอบคลุมระบบที่มีความสามารถในการปฏิบัติงานหลากหลายผ่านกลไกต่างๆ ดังนี้

\textbf{ขอบเขตด้านเทคนิค} การศึกษาครอบคลุมหุ่นยนต์ที่สามารถปรับเปลี่ยนฟังก์ชันผ่าน (1) การปรับโครงสร้างทางกายภาพแบบโมดูลาร์ (2) การเปลี่ยนแปลงอัลกอริทึมควบคุมและซอฟต์แวร์ และ (3) การผสมผสานโมดูลฮาร์ดแวร์ที่แตกต่างกัน

\textbf{ขอบเขตด้านการประยุกต์ใช้} เน้นการใช้งานในสภาพแวดล้อมอุตสาหกรรมและการค้า รวมถึงการแพทย์ การสำรวจ และการบริการ โดยไม่รวมถึงหุ่นยนต์เฉพาะทางที่ไม่สามารถปรับเปลี่ยนฟังก์ชันได้

\textbf{ขอบเขตด้านเวลา} การศึกษาเน้นงานวิจัยและพัฒนาตั้งแต่ปี ค.ศ. 2015 ถึงปัจจุบัน โดยเฉพาะอย่างยิ่งความก้าวหน้าในช่วง 5 ปีล่าสุดที่มีการประยุกต์ใช้ปัญญาประดิษฐ์และเทคโนโลยีการเรียนรู้ของเครื่อง

\subsection{คำจำกัดความสำคัญ}

เพื่อความชัดเจนในการศึกษา จึงกำหนดคำจำกัดความของแนวคิดสำคัญ ดังนี้

\textbf{หุ่นยนต์อเนกประสงค์ (Polyfunctional Robots)} หมายถึง ระบบหุ่นยนต์ที่สามารถปฏิบัติงานที่หลากหลายและแตกต่างกันได้ภายในระบบเดียว โดยมีความสามารถในการปรับเปลี่ยนฟังก์ชันการทำงานตามความต้องการของงานแต่ละประเภท

\textbf{หุ่นยนต์โมดูลาร์ (Modular Robots)} หมายถึง หุ่นยนต์ที่ประกอบด้วยโมดูลแยกส่วนที่สามารถเชื่อมต่อและแยกออกจากกันได้ เพื่อสร้างโครงสร้างและฟังก์ชันใหม่ตามต้องการ \parencite{bi2016survey}

\textbf{หุ่นยนต์ปรับโครงสร้างได้ (Self-Reconfiguring Robots)} หมายถึง หุ่นยนต์ที่สามารถเปลี่ยนแปลงรูปร่างและโครงสร้างของตนเองได้โดยอัตโนมัติ เพื่อให้เหมาะสมกับงานหรือสภาพแวดล้อมที่แตกต่างกัน \parencite{hameed2017modular}

\textbf{การควบคุมแบบลำดับชั้น (Hierarchical Control)} หมายถึง ระบบควบคุมที่จัดระดับการควบคุมเป็นชั้นๆ โดยชั้นบนมีหน้าที่วางแผนและตัดสินใจระดับสูง ส่วนชั้นล่างดำเนินการควบคุมรายละเอียดเฉพาะทาง

\subsection{ระเบียบวิธีการศึกษา}

การศึกษานี้ใช้วิธีการทบทวนวรรณกรรมเชิงพรรณนา (Descriptive Literature Review) ร่วมกับการวิเคราะห์เชิงเปรียบเทียบ โดยรวบรวมข้อมูลจากแหล่งข้อมูลทางวิชาการที่เชื่อถือได้ ประกอบด้วย

\textbf{แหล่งข้อมูลหลัก} วารสารวิชาการระดับนานาชาติที่ผ่านการประเมินโดยผู้ทรงคุณวุฒิ (Peer-reviewed Journals) เช่น IEEE Transactions on Robotics, International Journal of Robotics Research, และ Journal of Intelligent \& Robotic Systems

\textbf{แหล่งข้อมูลทุติยภูมิ} รายงานการประชุมวิชาการนานาชาติ (Conference Proceedings) มาตรฐานสากล และรายงานวิจัยจากสถาบันชั้นนำ เช่น NIST และ ISO

\textbf{เกณฑ์การคัดเลือกข้อมูล} เน้นงานวิจัยที่ตีพิมพ์ในช่วงปี ค.ศ. 2015-2025 มีการอ้างอิงและความน่าเชื่อถือสูง และเกี่ยวข้องโดยตรงกับหุ่นยนต์อเนกประสงค์หรือแนวคิดที่เกี่ยวข้อง

การวิเคราะห์ข้อมูลใช้การสังเคราะห์เชิงพรรณนา (Narrative Synthesis) โดยจัดกลุ่มข้อมูลตามประเด็นหลัก วิเคราะห์แนวโน้มและความสัมพันธ์ และสรุปเป็นองค์ความรู้ที่เป็นระบบ

\subsection{ประโยชน์ที่คาดว่าจะได้รับ}

การศึกษานี้คาดว่าจะให้ประโยชน์แก่ผู้เกี่ยวข้องหลายกลุ่ม ดังนี้

\textbf{สำหรับนักวิจัยและนักวิชาการ} เป็นการสังเคราะห์องค์ความรู้ที่เป็นปัจจุบันและครอบคลุม สามารถใช้เป็นฐานข้อมูลสำหรับการวิจัยต่อยอดในอนาคต

\textbf{สำหรับผู้ประกอบการและวิศวกร} ให้ข้อมูลสำคัญสำหรับการตัดสินใจลงทุนและการประยุกต์ใช้เทคโนโลยีหุ่นยนต์อเนกประสงค์ในภาคอุตสาหกรรม

\textbf{สำหรับนักศึกษาและผู้ที่สนใจ} เป็นแหล่งข้อมูลการเรียนรู้ที่เป็นระบบเกี่ยวกับเทคโนโลยีหุ่นยนต์ขั้นสูงและแนวโน้มการพัฒนาในอนาคต

\textbf{สำหรับหน่วยงานกำกับดูแล} ให้ข้อมูลประกอบการพิจารณาจัดทำนโยบายและมาตรฐานที่เกี่ยวข้องกับการใช้งานหุ่นยนต์อเนกประสงค์อย่างปลอดภัยและมีประสิทธิภาพ

\section{เนื้อหา}
% Section 2 content (เนื้อหา)

\section{การวิเคราะห์และอภิปราย}
% Section 3 content (การวิเคราะห์และอภิปราย)

\section{เอกสารประมิณขายงานฉบับสมบูรณ์}
% Section 4 content (เอกสารประมิณ - Assessment criteria)

\section{ความคิดสร้างสรรค์และความเยียบร้อย}
% Section 5 content (ความคิดสร้างสรรค์)

\section{สรุป}
% Section 6 content (สรุป)

% ----- Bibliography -----
\printbibliography[title=เอกสารอ้างอิง]

% ----- Appendices -----
\appendix

\section{ภาคผนวก ก: คำศัพท์เทคนิค}
% Technical glossary

\section{ภาคผนวก ข: ตารางเปรียบเทียบเทคโนโลยี}
% Technology comparison tables

\end{document}