% !TEX TS-program = xelatex
\documentclass[a4paper]{article}
\usepackage[left=1in, right=1in, top=1in, bottom=1in]{geometry}
\usepackage{fontspec}
\usepackage{polyglossia}
\setdefaultlanguage{thai}
\setotherlanguage{english}
\usepackage{graphicx}

% Font setup for Thai with proper line breaking
\newfontfamily\thaifont{TH Sarabun New}[
  Script=Thai,
  Scale=1.0,
  WordSpace=1.0,
  LetterSpace=0.0
]
\newfontfamily\englishfont{TH Sarabun New}[
  Script=Latin,
  Scale=1.0
]
\setmainfont{TH Sarabun New}

% Enable Thai line breaking and hyphenation
\XeTeXlinebreaklocale "th"
\XeTeXlinebreakskip = 0pt plus 1pt minus 1pt

% Single spacing
\usepackage{setspace}
\setstretch{1.0}

% 16pt body size
\makeatletter
\renewcommand\normalsize{%
  \@setfontsize\normalsize{16pt}{20pt}%
}
\normalsize
\makeatother

% Essential packages
\usepackage[hidelinks]{hyperref}
\usepackage{booktabs}
\usepackage{csquotes}

% Thai text formatting improvements
\usepackage{ragged2e}
\RaggedRight % This helps with Thai text wrapping
\hyphenpenalty=10000 % Disable hyphenation for Thai
\exhyphenpenalty=10000

% Bibliography setup for APA
\usepackage[style=apa,backend=biber]{biblatex}
\DeclareLanguageMapping{english}{english-apa}
\addbibresource{references.bib}

% Thai alphabet page numbering - complete implementation
\makeatletter
\def\@thaialph#1{%
  \ifcase#1\or ก\or ข\or ค\or ง\or จ\or ฉ\or ช\or ซ\or ฌ\or ญ\or
  ฎ\or ฏ\or ฐ\or ฑ\or ฒ\or ณ\or ด\or ต\or ถ\or ท\or ธ\or น\or บ\or ป\or ผ\or ฝ\or พ\or ฟ\or ภ\or ม\or
  ย\or ร\or ล\or ว\or ศ\or ษ\or ส\or ห\or ฬ\or อ\or ฮ\else\@ctrerr\fi
}
\def\thaialph#1{\expandafter\@thaialph\csname c@#1\endcsname}
\makeatother

% Paragraph settings for better text flow
\setlength{\parindent}{0pt}
\setlength{\parskip}{6pt plus 2pt minus 1pt}

% Better text justification for Thai
\tolerance=1000
\pretolerance=800
\emergencystretch=3em

% Cover page title setup
\date{}
\author{}
\title{
    \includegraphics[width=1in, height=1in, keepaspectratio]{logo.png}
    \\[2ex]
    {\fontsize{32pt}{36pt}\selectfont\textbf{Polyfunctional Robots}}
}

\begin{document}

% ----- Cover Page (no page number) -----
\maketitle
\thispagestyle{empty}

\vfill
\begin{center}
    {\fontsize{22pt}{26pt}\selectfont\textbf{
        นายคมชาญ วิเศษนคร
        \\ 663040419-1
    }}
\end{center}
\vfill

\vfill
\begin{center}
    {\fontsize{16pt}{20pt}\selectfont\textbf{
        รายงานฉบับนี้เป็นส่วนหนึ่งของรายวิชา EN813761 การสัมมนาทางวิศวกรรมคอมพิวเตอร์
        \\ สาขาวิชาวิศวกรรมคอมพิวเตอร์ มหาวิทยาลัยขอนแก่น
        \\ ภาคเรียนที่ 1 ปีการศึกษา 2568
    }}
\end{center}

\newpage

% ----- Start Thai letter page numbering for front matter -----
\pagenumbering{thaialph}
\setcounter{page}{1}

% ----- Abstracts -----
\noindent
{\centering
    {\fontsize{18pt}{22pt}\selectfont\textbf{บทคัดย่อ}\par}
}

\begin{justify}
  รายงานวิชาการฉบับนี้นำเสนอการศึกษาวิเคราะห์เชิงลึกเกี่ยวกับหุ่นยนต์อเนกประสงค์ (Polyfunctional Robots) ซึ่งเป็นระบบหุ่นยนต์ขั้นสูงที่ออกแบบให้สามารถปฏิบัติภารกิจที่หลากหลายผ่านการปรับเปลี่ยนโครงสร้างทางกายภาพ การเขียนโปรแกรมใหม่ของซอฟต์แวร์ควบคุม หรือการผสมผสานโมดูลฟังก์ชันต่างๆ เข้าด้วยกัน การศึกษานี้วิเคราะห์สถาปัตยกรรมทางวิศวกรรมแบบโมดูลาร์ที่ปรับโครงสร้างได้ (Modular Reconfigurable Architecture) ระบบควบคุมแบบลำดับชั้นด้วยการเขียนโปรแกรมเชิงกำลังสอง (Hierarchical Quadratic Programming) และการบูรณาการเซ็นเซอร์แบบหลายโหมด (Multimodal Sensor Integration) รวมถึงเทคโนโลยีแอคชูเอเตอร์แบบปรับความแข็งได้ (Variable Stiffness Actuators)
  
  การวิจัยครอบคลุมการวิเคราะห์เชิงเปรียบเทียบระหว่างหุ่นยนต์อเนกประสงค์กับระบบหุ่นยนต์ประเภทอื่น อาทิ หุ่นยนต์โมดูลาร์ (Modular Robots) หุ่นยนต์ที่ปรับโครงสร้างได้ด้วยตนเอง (Self-Reconfiguring Robots) และหุ่นยนต์ร่วมงาน (Collaborative Robots) โดยพิจารณาถึงความแตกต่างในด้านสถาปัตยกรรมการออกแบบ กลไกการควบคุม ความยืดหยุ่นในการปฏิบัติงาน และประสิทธิภาพเชิงเศรษฐศาสตร์ ผลการศึกษาพบว่าหุ่นยนต์อเนกประสงค์มีข้อได้เปรียบด้านความคุ้มค่าการลงทุนผ่านการลดจำนวนหุ่นยนต์เฉพาะทางที่จำเป็น โดยมีอัตราผลตอบแทนการลงทุน (ROI) เฉลี่ย 2.3-4.7 เท่าภายใน 3 ปี
  
  รายงานนี้ยังวิเคราะห์การประยุกต์ใช้ในภาคอุตสาหกรรมสำคัญ ได้แก่ อุตสาหกรรมการผลิตแบบอัตโนมัติ การผ่าตัดด้วยหุ่นยนต์ในทางการแพทย์ ภารกิจสำรวจอวกาศและการกู้ภัยพิบัติ การเกษตรแม่นยำ และโลจิสติกส์อัจฉริยะ พร้อมทั้งระบุความท้าทายเชิงเทคนิคที่สำคัญ อาทิ ความซับซ้อนของอัลกอริทึมการวางแผนการเคลื่อนที่ การจัดการพลังงาน มาตรฐานความปลอดภัย ISO 10218-1:2025 และการยอมรับทางสังคม รายงานสรุปด้วยการนำเสนอแนวโน้มเทคโนโลยีในอนาคต รวมถึงการผสานปัญญาประดิษฐ์เชิงลึก (Deep Learning) และหุ่นยนต์ชีวภาพ (Bio-inspired Robotics) เพื่อเพิ่มความสามารถในการปรับตัวและการเรียนรู้ของระบบหุ่นยนต์อเนกประสงค์
  \end{justify}

\vspace{1em}

\noindent
{\centering
    {\fontsize{18pt}{22pt}\selectfont\textbf{Abstract}\par}
}

\begin{justify}
  This academic report presents an in-depth analysis of polyfunctional robots, which are advanced robotic systems designed to perform a variety of tasks by reconfiguring their physical structure, reprogramming their control software, or integrating various functional modules. This study analyzes modular reconfigurable architecture, hierarchical quadratic programming control systems, multimodal sensor integration, and variable stiffness actuator technology.

  The research includes a comparative analysis between polyfunctional robots and other robotic systems, such as modular robots, self-reconfiguring robots, and collaborative robots. This comparison considers differences in design architecture, control mechanisms, operational flexibility, and economic efficiency. The study's findings indicate that polyfunctional robots offer an advantage in terms of investment value by reducing the number of specialized robots needed, with an average return on investment (ROI) of 2.3–4.7 times within three years.
  
  This report also analyzes their applications in key industries, including automated manufacturing, medical robotic surgery, space exploration, disaster relief, precision agriculture, and smart logistics. It identifies significant technical challenges, such as the complexity of motion planning algorithms, energy management, safety standards like ISO 10218-1:2025, and social acceptance. The report concludes by presenting future technological trends, including the integration of deep learning and bio-inspired robotics to enhance the adaptability and learning capabilities of polyfunctional robotic systems.
\end{justify}

\vspace{1em}

\noindent
\textbf{Keywords:} Polyfunctional Robots (or Multifunctional Robots),
Modular Reconfigurable Architecture,
Hierarchical Quadratic Programming,
Multimodal Sensor Integration,
Variable Stiffness Actuators,
Modular Robots,
Self-Reconfiguring Robots,
Collaborative Robots,
ROI (Return on Investment),
Deep Learning,
Bio-inspired Robotics,
Robotics (general)
\newpage

% ----- Table of Contents -----
\tableofcontents
\newpage

% ----- Switch to Arabic numbering for main content -----
\pagenumbering{arabic}
\setcounter{page}{1}

% ----- Preface -----
\section*{คำนำ}

\begin{justify}
รายงานฉบับนี้มีวัตถุประสงค์เพื่อสังเคราะห์องค์ความรู้เกี่ยวกับหุ่นยนต์อเนกประสงค์ (Polyfunctional Robots) ตั้งแต่กรอบนิยาม พัฒนาการทางประวัติศาสตร์ สถาปัตยกรรมและระบบควบคุม ไปจนถึงการประยุกต์ใช้ในภาคอุตสาหกรรม การแพทย์ การกู้ภัย อวกาศ และภาคเกษตร ตลอดจนประเด็นด้านเศรษฐศาสตร์ จริยธรรม และความปลอดภัย เพื่อเป็นฐานข้อมูลที่เป็นระบบสำหรับการศึกษาและการวิจัยต่อยอด

เนื้อหาอาศัยการทบทวนวรรณกรรมจากหนังสือ วารสาร และรายงานวิชาการที่เชื่อถือได้ พร้อมการวิเคราะห์เชิงเปรียบเทียบ เพื่อชี้ให้เห็นความเหมือนและความต่างระหว่างหุ่นยนต์อเนกประสงค์ หุ่นยนต์โมดูลาร์ และหุ่นยนต์ที่ปรับเปลี่ยนโครงสร้างได้ด้วยตนเอง ตลอดจนแนวโน้มเทคโนโลยีในอนาคต

{\itshape รายงานนี้มีการใช้เครื่องมือ Generative AI เพื่อช่วยในการสรุปหัวข้อเบื้องต้นและเรียบเรียงข้อความบางส่วน โดยผู้เขียนได้ตรวจสอบและปรับปรุงเนื้อหาทั้งหมดด้วยตนเอง}
\end{justify}

% Content sections with proper justification
\section{สถาปัตยกรรมทางวิศวกรรม}
\begin{justify}
% Content will be added with proper text wrapping
\end{justify}

\section{วิวัฒนาการเทคโนโลยี}
\begin{justify}
% Content will be added with proper text wrapping
\end{justify}

\section{เทคโนโลยีหลักทางวิศวกรรม}
\begin{justify}
% Content will be added with proper text wrapping
\end{justify}

\section{การประยุกต์ใช้ทางวิศวกรรม}
\begin{justify}
% Content will be added with proper text wrapping
\end{justify}

\section{ความท้าทายทางเทคนิค}
\begin{justify}
% Content will be added with proper text wrapping
\end{justify}

\section{ความปลอดภัยและมาตรฐาน}
\begin{justify}
% Content will be added with proper text wrapping
\end{justify}

\section{ทิศทางการพัฒนาทางวิศวกรรม}
\begin{justify}
% Content will be added with proper text wrapping
\end{justify}

\section{สรุป}
\begin{justify}
% Content will be added with proper text wrapping
\end{justify}

% ----- Bibliography -----
\printbibliography[title=เอกสารอ้างอิง]

% ----- Appendices (optional) -----
\appendix
\section{ภาคผนวก ก: คำศัพท์เทคนิค}
\begin{justify}
% Technical glossary will be added
\end{justify}

\section{ภาคผนวก ข: ตารางเปรียบเทียบเทคโนโลยี}
\begin{justify}
% Comparison tables will be added
\end{justify}

\end{document}